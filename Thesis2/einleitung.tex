\section{Einleitung}

Einleitung und Fragestellung\\
\cite{houghpatent}

\subsection{Grabung Kapitol}
Grabungsverlauf bis 2014 (recherchieren)\\
Übernahme durch DAI (recherchieren)\\

\subsection{Datensatz vorstellen}
Herkunft\\
Umfang\\
Fragestellungen des Projektes\\


%\subsubsection{Tafelvergleiche}

%Im Rahmen der Arbeit wurden weitere Tafeln exemplarisch dem Algorithmus unterzogen. Dabei handelte es sich um Aufnahmen der späteren Grabungen des Deutschen Archäologischen Institutes am Kapitol in Rom sowie um vergleichbare Fotos von Bodenuntersuchungen der Gruppe Terrestrische Ökohydrologie der Friedrich-Schiller-Universität Jena. Der ursprüngliche Gedanke dahinter war eine möglichst universale Detektion von Tafeln aller Art anzustreben. Die unterschiedlichen Daten konnten dabei vor allem Stärken und Schwächen der letztlich gewählten Technik aufzeigen.\\
%Die Tafeln beider Projekte sollen im Folgenden kurz vorgestellt werden, um das Spektrum der Komplexität 
%evtl. Vergleiche zu Tafeln aus späterer Grabung als Positivbeispiel:\\
%besser gearbeitete Tafeln\\
%besser lesbare Schrift\\
%evtl. Vergleiche zu Tafeln der Bodenkunde als Negativbeispiel:\\
%Tafel schwierig durch Form und Farbe\\
%Klarsichthülle: Reflektion und Formveränderung\\
%oft verdeckt\\
%Bilder zur Veranschaulichung einfügen

%\subsubsection{Schrift}
%Kreide auf Schiefer
%Probleme wie Handschrift, Verwischung, Karomuster
%\subsection{Pipeline}
%Struktur der Arbeit wie Pipeline:
%Bildakquise
%Objekterkennung
%Crop-Verfahren
%Pre-Processing
%OCR
%Evaluation
%Ergebnis