%Zusammenfassung Ausgangslage
%Zusammenfassung Vorgehen
%Zusammenfassung Ergebnisse
%Ausblick
\section{Fazit}

In dieser Arbeit konnte gezeigt werden, dass das grundsätzliche Prinzip, Tafeln auf Grabungsfotos mit Methoden der Computer Vision zu erkennen, funktioniert. Weiter konnte gezeigt werden, dass das Auslesen durch eine OCR-Engine, in diesem Fall Tesseract, ebenfalls möglich ist. Problematisch erwiesen sich Faktoren wie die Lage der Tafel, die Beleuchtung und die von Hand ausgeführte Beschriftung der Tafeln mit Kreide. Bei der Texterkennung erwies sich Letzteres als Hauptursache für schlechte Ergebnisse.
Für eine Automatisierung des Prozesses, die Metadaten aus den Tafeln zu extrahieren, sind die Ergebnisse nicht ausreichend. Durch die Selektion der Fotos mit Tafeln sowie das Bereitstellen des erkannten Textes, der im Anschluss manuell korrigiert wird, kann der Arbeitsaufwand jedoch deutlich reduziert werden. Zudem konnten Empfehlungen abgegeben werden, wie Fotografieren zur Metadatenerfassung mittels Tafeln in Zukunft durchgeführt werden kann, um bessere Ergebnisse zu erzielen.
In der Zukunft könnte der Detektionsprozess optimiert werden. So hat die Erkennung von Tafeln, deren Rand teilweise verdeckt ist, Verbesserungspotential. Weitere Untersuchungen zu anderen Tafeln und Datensätzen wären lohnenswert.