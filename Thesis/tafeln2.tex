\section{Objektdetektion}

Das folgende Kapitel befasst sich mit den ersten beiden Schritten in der automatisierten Analyse der Grabungsfotos: der Erkennung der Schiefertafeln und ihrer Extraktion aus dem Gesamtbild.\\
Zunächst werden verschiedene Möglichkeiten der Erkennung präsentiert und diskutiert. Der Schwerpunkt liegt hier auf dem schlussendlich umgesetzten Verfahren.
Abschließend wird das mit der Tafeldetektion verbundene Ausschneiden der gefundenen Tafeln aus dem Gesamtbild vorgestellt.

\subsection{Detektionsverfahren}

Aus den oben genannten Punkten und den Anforderungen der Aufgabenstellung lässt sich als erster Arbeitsschritt die Detektion der Tafeln ableiten. Die wichtigste Prämisse ist dabei, dass Falsch-Negative, also nicht erkannte Tafeln, vermieden werden. Grund dafür ist, dass diese für die weitere Bearbeitung komplett verloren sind. Falsch-Positive sollten ebenfalls weitestgehend ausgeschlossen werden. Diese können jedoch in späteren Bearbeitungsschritten, bspw. der Texterkennung, erkannt und aussortiert werden. Daher ist dieses Kriterium von niedrigerer Priorität.
Dieses Kapitel wird sich daher verschiedenen Verfahren widmen, mit denen rechteckige, beschriftete Objekte erkannt werden können. Diese Verfahren werden kurz vorgestellt und ihre Ergebnisse in einer ersten, explorativen Umsetzung präsentiert. Anschließend wird begründet, warum diese Verfahren im Rahmen des Projektes nicht nicht zum Einsatz kommen. Final wird dann das Kontur-basierte Erkennungsverfahren erläutert, mit dem die besten Ergebnisse erzielt wurden.

%das Kapitel wird aus einer Kurzvorstellung der einzelnen Ansätze bestehen, die in Frage kamen und ausprobiert wurden
%In der Tiefe wird sich anschließend mit dem letztlich umgesetzten Verfahren, basierend auf \verb|cv2.contours|, auseinandergesetzt.

\subsubsection{Feature Detection}

was ist feature detection?
wie funktioniert sie?
was war die Idee hinter dem Ansatz?
Wie sehen die Ergebnisse aus?

\subsubsection{CNN}

hier werden CNNs vorgestellt
was sind CNNs?
Was können sie, wie funktionieren sie?
Warum habe ich sie ausprobiert, was war die Idee dahinter?
Warum habe ich nicht selbst trainiert?
Wie sehen die Ergebnisse aus?

\subsubsection{Contours}

was sind contours?
worin besteht die Grundidee?
wie wurde diese Idee umgesetzt? -> Flowchart
Zweispurigkeit der Ansätze: iterativ und adaptiv. Erklären warum.
\verb|rect_detect| als Finale, in dem die beiden Ansätze wieder zusammengeführt werden

\subsection{Cropverfahren}

Was ist die Aufgabe beim Crop?
Worin liegen hier die Schwierigkeiten?
Auch hier wieder Zweispurigkeit der Ansätze erklären

\subsubsection{simple crop}

Was ist die Idee?
Wie wurde sie umgesetzt?
Wo liegen die Probleme?

\subsubsection{Hough}

Was ist die Idee?
Wie wurde sie umgesetzt?
Wo liegen die Probleme?

\subsection{Zusammenfassung}
evtl zusammenfassen wie vorgegangen wurde, warum dieser Weg gut ist und was das Wichtigste Ergebnis ist