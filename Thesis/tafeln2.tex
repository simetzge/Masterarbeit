\section{Tafeldetektion}

Das folgende Kapitel befasst sich mit dem ersten Schritt in der automatisierten Analyse der Grabungsfotos: der Erkennung der Schiefertafeln.\\
Zunächst sollen die Tafeln vorgestellt und die Probleme bei der Detektion erörtert werden. Im Anschluss werden verschiedene Möglichkeiten der Erkennung präsentiert und diskutiert. Der Schwerpunkt liegt hier auf dem schlussendlich umgesetzten Verfahren.
Abschließend wird das mit der Tafeldetektion verbundene Ausschneiden der gefundenen Tafeln aus dem Gesamtbild vorgestellt.

\subsection{Die Tafeln}

Die Verwendung von Tafeln zur Dokumentation von Fund- und Grabungsarealen ist in allen, im weitesten Sinne grabenden, Wissenschaften weit verbreitet (Vgl. Bildquellen) . So setzt auch die Archäologie diese Methode ein. Dabei werden neben den zu dokumentierenden Gebieten verschiedenste Formen von Tafeln oder Schildern platziert, auf denen Zeit und Ort der Aufnahme sowie weitere bild- und motivbezogene Informationen festgehalten werden können. Der Vielfalt von Form und Material der Tafeln ist dabei keine Grenze gesetzt.

\subsubsection{Die Tafeln als Ausgangsmaterial}

Wie sehen die Tafeln aus
was kann bei der Erkennung nützlich sein, wo liegen Probleme und Schwierigkeiten (Bebildern)

\subsubsection{Tafelvergleiche}

Im Rahmen der Arbeit wurden weitere Tafeln exemplarisch dem Algorithmus unterzogen. Dabei handelte es sich um Aufnahmen der späteren Grabungen des Deutschen Archäologischen Institutes am Kapitol in Rom sowie um vergleichbare Fotos von Bodenuntersuchungen der Gruppe Terrestrische Ökohydrologie der Friedrich-Schiller-Universität Jena. Der ursprüngliche Gedanke dahinter war eine möglichst universale Detektion von Tafeln aller Art anzustreben. Die unterschiedlichen Daten konnten dabei vor allem Stärken und Schwächen der letztlich gewählten Technik aufzeigen.\\
Die Tafeln beider Projekte sollen im Folgenden kurz vorgestellt werden, um das Spektrum der Komplexität 
evtl. Vergleiche zu Tafeln aus späterer Grabung als Positivbeispiel:\\
besser gearbeitete Tafeln\\
besser lesbare Schrift\\
evtl. Vergleiche zu Tafeln der Bodenkunde als Negativbeispiel:\\
Tafel schwierig durch Form und Farbe\\
Klarsichthülle: Reflektion und Formveränderung\\
oft verdeckt\\
Bilder zur Veranschaulichung einfügen

\subsection{Detektierungsverfahren}

das Kapitel wird aus einer Kurzvorstellung der einzelnen Ansätze bestehen, die in Frage kamen und ausprobiert wurden
In der Tiefe wird sich dann mit dem letztlich umgesetzten Verfahren, basierend auf contours, auseinandergesetzt

\subsubsection{Feature Detection}

was ist feature detection?
wie funktioniert sie?
was war die Idee hinter dem Ansatz?
Wie sehen die Ergebnisse aus?

\subsubsection{CNN}

hier werden CNNs vorgestellt
was sind CNNs?
Was können sie, wie funktionieren sie?
Warum habe ich sie ausprobiert, was war die Idee dahinter?
Warum habe ich nicht selbst trainiert?
Wie sehen die Ergebnisse aus?

\subsubsection{Contours}

was sind contours?
worin besteht die Grundidee?
wie wurde diese Idee umgesetzt? -> Flowchart
Zweispurigkeit der Ansätze: iterativ und adaptiv. Erklären warum.
rect_detect als Finale, in dem die beiden Ansätze wieder zusammengeführt werden

\subsection{Cropverfahren}

Was ist die Aufgabe beim Crop?
Worin liegen hier die Schwierigkeiten?
Auch hier wieder Zweispurigkeit der Ansätze erklären

\subsubsection{simple crop}

Was ist die Idee?
Wie wurde sie umgesetzt?
Wo liegen die Probleme?

\subsubsection{Hough}

Was ist die Idee?
Wie wurde sie umgesetzt?
Wo liegen die Probleme?

\subsection{Zusammenfassung}
evtl zusammenfassen wie vorgegangen wurde, warum dieser Weg gut ist und was das Wichtigste Ergebnis ist