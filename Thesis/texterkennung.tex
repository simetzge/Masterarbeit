\section{Texterkennung}
\subsection{Texterkennung allgemein}
Ursprünge der Texterkennung\\
Wechsel auf machine learning\\
Schwerpunkte\\
Handschrift vs. Druckschrift\\

\subsection{Das Ausgangsmaterial}
Probleme bennenen wie: verwischte Kreide, das Karomuster der Tafeln, Handschrift per se, Licht und Beleuchtung\\
Unterschiedliche Handschrift, unterschiedliche Struktur der Bezeichnung (z.B. fehlendes ''US'', komplexe Beschriftung etc.)\\

\subsection{Tesseract}

evtl ALternativen\\
Tesseract: was kann es, wie funktioniert es\\
inzwischen Entwicklung durch Google\\
Wechsel von Charactererkennung via CV zu Zeilenerkennung via machine learning\\
wie funktioniert die Box-detection?\\

\subsection{Vorgehen}
Evaluation: Vorgehen, Überlegungen\\
Preprocessing: besondere Herausforderungen, vorgehen, beide Varianten vorstellen\\
normales Modell\\
eigenes Modell\\
Vergleich: Tafeln aus späterer Grabung (gesetzte Lettern)\\
evtl. Vergleich Tafeln Bodenkunde\\