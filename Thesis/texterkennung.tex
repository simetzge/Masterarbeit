\section{Texterkennung}
\subsection{Texterkennung allgemein}
Tesseract: was kann es, wie funktioniert es\\
\subsection{Das Ausgangsmaterial}
Probleme bennenen wie: verwischte Kreide, das Karomuster der Tafeln, Handschrift per se, Licht und Beleuchtung\\

Funktion der Tafeln und Beschreibung\\
rechteckig -> klare Umrisse\\
doppelter Rand\\
grobe Verarbeitung\\
unterschiedliche Positionierung (Entfernung, Winkel)\\
wechselnde Beschriftung\\
wechselnde Beleuchtung\\
rechteckiges Muster im Hintergrund\\

Problematiken aufzeigen: unterschiedliche Größe, Licht und Beleuchtung, Winkel, Verdeckung\\
andere Objekte die als Tafeln erkannt werden könnten\\
\subsection{Tesseract}

evtl ALternativen\\

\subsection{Vorgehen}
Evaluation: Vorgehen, Überlegungen\\
Preprocessing: besondere Herausforderungen, vorgehen, beide Varianten vorstellen\\
normales Modell\\
eigenes Modell\\
Vergleich: Tafeln aus späterer Grabung (gesetzte Lettern)\\
evtl. Vergleich Tafeln Bodenkunde\\