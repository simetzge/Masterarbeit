\section{Einleitung}

%Einleitung und Fragestellung

In der Archäologie, wie auch in anderen, \glqq grabenden\grqq{} Wissenschaften, werden Fotografien zur Dokumentation von Grabungen und Befunden eingesetzt. Diese werden durch die Positionierung von beschrifteten Tafeln im Bild mit Metadaten angereichert. Sowohl für die Digitalisierung von Altbeständen als auch für die Verwaltung von bereits digital aufgenommenen Fotos ist es notwendig, diese Metadaten auszulesen und in Datenbanken oder in die Metadaten der Bilddateien selbst einzupflegen.\\
Diese Aufgabe stellt sich auch im Kapitol-Projekt am Deutschen Archäologischen Institut in Rom. Gegenstand des Projektes sind Grabungen auf dem Gebiet des historischen römischen Staatsheiligtums unter Einbeziehung der in der Vergangenheit bereits durchgeführten Grabungen. Es liegen zahlreiche Fotos vor, die Metadaten in Form von Schrifttafeln enthalten. In der Auseinandersetzung mit diesem Material stellt sich die Frage, ob der Aufwand durch eine automatisierte Erkennung der Tafeln und deren Beschriftung reduziert werden kann. Erforderlich wären dazu zwei Schritte: Die Lokalisation der Tafeln und die Erkennung des Textes darauf.\\
% Einführung Tafelerkennung: Shape detection, Objekterkennung CNNs 
Die Detektion bestimmter Formen und Objekte gehört zu den klassischen Anwendungen der Computer Vision. Die Verwendung der Konturen zur Erkennung eines Objektes ist erprobt und entsprechende Funktionen sind fester Bestandteil der Computer Vision-Bibliothek OpenCV. Grundlegend für diese Anwendungsbereiche sind Kanten- und Linienerkennung wie die von Canny \cite{cannyedge} bzw. Hough \cite{houghpatent}{} so wie Template-Matching-Verfahren \cite{opencvtemplatematching}{} zur Identifikation von Bildausschnitten in einem größeren Gesamtbild. Die praktischen Anwendungsbereiche sind vielfältig, so z.B. in der Mustererkennung \cite{adamek}{} oder als Orientierungshilfe für autonome Roboter \cite{shaw}{}. Auch wenn für komplexere Aufgaben inzwischen vor allem auf Neuronale Netzwerke zurückgegriffen wird \cite{introduction}{}, werden Objekte wie AR-Marker und QR-Codes auch heute noch mit diesen Methoden erfasst \cite{armarker}{}.\\
%Einführung Texterkennung: Feature detection bis Neural Networks, Herausforderungen vor allem Handschrift und natural scenes, selten beides zusammen
Auch an und mit der Optical Character Recognition (OCR) wird seit der Entwicklung digitaler Computer gearbeitet. Die ersten kommerziellen Produkte waren bereits in den 50er Jahren verfügbar \cite{eikvil}{} und werden seitdem stetig verbessert. Der letzte große Schritt war die Einführung von Neuronalen Netzwerken auch in diesem Bereich. Die im Rahmen dieser Arbeit verwendete Tesseract-OCR-Engine vollzog diesen Schritt 2018  \cite{tesseractrelease}{}.\\
In der OCR wird zwischen offline- und online-Texterkennung unterschieden, wobei online das Tracking des Schreibprozesses meint und sich offline-Texterkennung mit bereits geschriebenem Text befasst \cite{feldmann}{}. Im Bereich der offline-Erkennung wird unterschieden, ob der Text als Scan oder vergleichbares Ausgangsmaterial in einer kontrollierten Umgebung vorliegt, oder ob es sich um Text in einer natürlichen Umgebung (\textit{natural scenes}{}) handelt. Bei Letzterem muss der Text erst gefunden und Faktoren, wie der Winkel zur Kamera oder die Lichtverhältnisse, ausgeglichen werden müssen  \cite{forsberg}{} \cite{qixiangye}{} \cite{xilinchen}. Diese Faktoren erschweren die Texterkennung deutlich, weshalb hier mit mehr Aufbereitung und schlechteren Ergebnissen gerechnet werden muss, als in natürlicher Umgebung.\\
Eine weitere große Herausforderung im Bereich offline-OCR ist die Erkennung von Handschriften. Handschriften verschiedener Personen unterscheiden sich stark voneinander  \cite{sumedhahallale}, aber auch die Handschrift einer Person ist unregelmäßig und variiert stark \cite{feldmann}. Druckbuchstaben können dabei von herkömmlichen \textit{Engines}{} deutlich besser erkannt werden als Schreibschrift. In den Geisteswissenschaften wird dieses Thema vor allem in Bezug auf historische Quellen interessant. Programme wie Transkribus \cite{transkribus}{} bieten die Möglichkeit, Texte zu transkribieren und auf dieser Basis ein Neuronales Netzwerk auf spezifische (Hand-)Schriften zu trainieren. Entscheidend für den Erfolg dieses Vorgehens ist das Vorliegen ausreichender Datenmengen; Eine Voraussetzung, die in diesem Projekt nicht erfüllt ist. 

%\subsection{Theorie}

\subsection{Grabung Kapitol}
%Grabungsverlauf bis 2014 (recherchieren)\\
%Übernahme durch DAI (recherchieren)\\
Seit den späten 90er Jahren führte die \textit{Sovraintendenza ai beni culturali del Comune di Roma}{} Grabungen im Gebiet des antiken römischen Staatsheiligtums auf dem Kapitolinischen Hügel in Rom durch \cite{danti}{}. In den Jahren 2011-2014 wurde verstärkt auch auf dem Gebiet um das ehemalige preußische Hospital gegraben. Seit 2018 wird das Projekt vom Deutschen Archäologischen Institut und den \textit{Musei Capitolini}{} weitergeführt \cite{kapitol}{}. Dabei sollen Fragen zur \glqq kulturgeschichtlichen Entwicklung des Kapitolhügels in Antike, Mittelalter und Neuzeit\grqq{} \cite{kapitol}{} beantwortet werden.
Im Rahmen des Projektes sollen digitale Methoden zu Dokumentation und Erkenntnisgewinn eingesetzt werden. Das bedeutet einerseits die Erfassung aktueller Befunde mittels Bodenradar, Fotogrammetrie oder 3D-Modellen \cite{kapitol}. Andererseits sollen historische Quellen und Dokumentationen von Altgrabungen digitalisiert und zugänglich gemacht werden. 
%Ziel dieser Maßnahmen ist es, ein möglichst umfassendes Bild des Kapitols zu unterschiedlichen Zeiten zu gewinnen.
Historische Karten, Dokumente und Beschreibungen sowie die bei bisherigen Grabungen gemachten Funde und Befunde könnten dabei helfen, das Bild des Kapitols zu vervollständigen -- durch die reine Erschließung, aber auch durch die Verarbeitung mit modernen Mitteln der Wissenschaft und Technik.
Dazu gehören auch die Grabungsfotos, die Gegenstand dieser Arbeit sind.
%\subsection{Datensatz}
%Herkunft\\
%Umfang\\
%Fragestellungen des Projektes\\



%\subsubsection{Tafelvergleiche}

%Im Rahmen der Arbeit wurden weitere Tafeln exemplarisch dem Algorithmus unterzogen. Dabei handelte es sich um Aufnahmen der späteren Grabungen des Deutschen Archäologischen Institutes am Kapitol in Rom sowie um vergleichbare Fotos von Bodenuntersuchungen der Gruppe Terrestrische Ökohydrologie der Friedrich-Schiller-Universität Jena. Der ursprüngliche Gedanke dahinter war eine möglichst universale Detektion von Tafeln aller Art anzustreben. Die unterschiedlichen Daten konnten dabei vor allem Stärken und Schwächen der letztlich gewählten Technik aufzeigen.\\
%Die Tafeln beider Projekte sollen im Folgenden kurz vorgestellt werden, um das Spektrum der Komplexität 
%evtl. Vergleiche zu Tafeln aus späterer Grabung als Positivbeispiel:\\
%besser gearbeitete Tafeln\\
%besser lesbare Schrift\\
%evtl. Vergleiche zu Tafeln der Bodenkunde als Negativbeispiel:\\
%Tafel schwierig durch Form und Farbe\\
%Klarsichthülle: Reflektion und Formveränderung\\
%oft verdeckt\\
%Bilder zur Veranschaulichung einfügen

%\subsubsection{Schrift}
%Kreide auf Schiefer
%Probleme wie Handschrift, Verwischung, Karomuster
%\subsection{Pipeline}
%Struktur der Arbeit wie Pipeline:
%Bildakquise
%Objekterkennung
%Crop-Verfahren
%Pre-Processing
%OCR
%Evaluation
%Ergebnis